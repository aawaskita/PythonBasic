\chapter{Interaksi berkas}
\section{Pendahuluan}
Lingkup interaksi di sini adalah interaksi dengan berkas teks (\texttt{ASCII}) untuk tujuan membaca, memformat ulang dalam bentuk penyajian di layar maupun disimpan kembali ke berkas teks. Selain teks, maka diperlukan proses tambahan untuk mengubahnya. Hal ini disebabkan karena umumnya data yang dianalisis dalam keilmuan data disimpan dalam bentuk teks dengan format csv (\textit{comma separated value}). Beberapa mungkin disimpan dalam aplikasi \textit{spreadsheet} seperti Microsoft Excel$\copyright$. 

Selain itu, kita mungkin saja terlibat dengan lebih dari satu berkas yang tersebar di \textit{sub directory}  yang berbeda. Atau bisa saja perlu menyusun ulang struktur \textit{directory} baru yang lebih mudah dipahami. Satu contoh kasus\footnote{\url{http://otmedia.lirmm.fr/LifeCLEF/PlantCLEF2017/}}, data yang berisi citra tumbuhan dari bagian yang berbeda seperti bunga, daun atau buah tidak tersusun berdasarkan bagian-bagian tersebut. Sementara kita memerlukan data tersebut tersusun berdasarkan bagian tumbuhan, bahkan mengikuti struktur \textit{family-genus-species}. Atau contoh lain\footnote{\url{https://www.unsw.adfa.edu.au/unsw-canberra-cyber/cybersecurity/ADFA-NB15-Datasets/bot\_iot.php}}, di mana dataset terpisah dalam berkas yang berbeda untuk kemudahan proses unduh. Sementara di masing-masing berkas terdapat data dari kelas-kelas yang berbeda. Untuk mengetahui proporsi data dari setiap kelas, kita perlu membaca semua potongan berkas yang tersedia.

\section{Berkas tunggal}
Sebagai bahan latihan, kita akan membaca dataset iris\footnote{\url{https://archive.ics.uci.edu/ml/machine-learning-databases/iris/iris.data}}. Pembacaan dilakukan baris per baris dan menampilkan isinnya. Dataset tersebut berisi informasi tentang dimensi bunga iris, yaitu \textit{sepal length}, \textit{sepal width}, \textit{petal length}, \textit{petal width} yang masing-masing dalam satuan \texttt{cm}. Kolom terakhir merupakan kelas dari 3 jenis bunga iris yang ada pada dataset tersebut, masing-masing Setosa, Versicolor dan Virginica. Data yang ditampilkan harus memiliki format 5 baris yang setiap barisnya adalah

\texttt{jenis bunga iris ke-i:}

\texttt{sepal length = $\ldots$ cm}

\texttt{sepal width = $\ldots$ cm}

\texttt{petal length = $\ldots$ cm}

\texttt{petal width = $\ldots$ cm}


Perhatikan \lstlistingname~\ref{lst:bacairis}. Program tersebut bertujuan untuk membaca data yang disimpan dalam bentuk tabular dan menampilkannya di layar dengan format baru. Simpan \lstlistingname~\ref{lst:bacairis} dalam berkas berekstensi \texttt{.py} lalu jalankan untuk melihat hasilnya dengan perintah \texttt{python nama\_file.py}.

\lstinputlisting[language=python, numbers=left, numberstyle=\small, caption=Membaca dataset iris dan menampilkan isinya dengan format tertentu, showstringspaces=false, label=lst:bacairis]{script/bacairis.py}

Berikut adalah penjelasan perintah pada \lstlistingname~\ref{lst:bacairis} berdasarkan urutan barisnya.
\begin{enumerate}
  \item Baris ke-1: membuat \texttt{pointer} ke berkas, yang dalam contoh ini adalah \texttt{iris.data} yang menjadi argumen pertama dari fungsi \texttt{open}. Sedangkan argumen keduanya adalah \texttt{'r'}, menunjukkan mode baca. Untuk mode tulis, gunakan argumen \texttt{'w'}.
  \item Baris ke-2: menyiapkan variabel \textit{dictionary} yang akan digunakan untuk menyimpan informasi kelas data berikut jumlahnya.
  \item Baris ke-3: menyiapkan variabel \texttt{int} yang akan digunakan untuk menyimpan informasi total data.
  \item Baris ke-4: memulai perulangan untuk membaca berkas per baris melalui perintah \texttt{a.readlines()}. Jika hanya diperlukan untuk membaca satu baris saja, gunakan perintah \texttt{a.readline()}. Baris yang berhasil dibaca akan disimpan dalam variabel \texttt{baris}.
  \item Baris ke-5 \& 18: penanganan kesalahan yang disebabkan oleh kondisi berkas yang baris terakhirnya kosong sehingga tidak dapat diolah seperti baris lain di atasnya.
  \item Baris ke-6: memisahkan baris yang dibaca berdasarkan pembatas (\textit{delimiter}) berupa karakter \texttt{,}. Hasilnya disimpan dalam \texttt{list} dengan nama \texttt{element}.
  \item Baris ke-7: membersihkan \texttt{element} terakhir dari karakter \escape{n}, dan menyimpannya dalan \texttt{list} dengan nama \texttt{kelas}. Nilai dari \texttt{kelas} yang akan digunakan disimpan di \texttt{kelas[0]}.
  \item Baris ke-8 s/d 11: blok kondisi yang ketika nilai kelas belum ada di variabel \texttt{jenis}, ia akan ditambahkan sebagai \texttt{key} dengan nilai \texttt{1}. Sedangkan jika sudah menjadi salah satu \texttt{key}, maka nilainya ditambah \texttt{1}.
  \item Baris ke-12 s/d 16: menampilkan isi dari baris yang sedang dibaca, yang sebelumnya telah dipisah-pisah ke layar. Sejatinya, perintah \texttt{print} menerima argumen \texttt{str}. Jika ada karakter/kata eksplisit akan ditampilkan bersama karakter/kata yang disimpan pada variabel, maka penulisannya perlu dibedakan, dan dihubungkan dengan operator \texttt{+}. Dari sini terlihat bahwa operator \texttt{+} tidak hanya dapat digunakan dalam operasi aritmatika. Khusus untuk variabel yang akan ditampilkan dengan perintah \texttt{print}, tetapi belum dalam bentuk \texttt{str} (seperti pada baris ke-12), maka perlu dilakukan transformasi. Perintah yang digunakan adalah \texttt{str}, sebagai target jenis variabel yang diperlukan. Jika pada kondisi tertentu, diperlukan untuk mengubah jenis variabel ke \texttt{int}, maka digunakan perintah yang sama dengan target jenis variabelnya, dalam hal ini \texttt{int}.
  \item Baris ke-17: mengakumulasi total data dari berkas yang dibaca.
  \item Baris ke-19: mendefinisikan perintah ketika baris yang dibaca tidak dapat diolah dengan perintah yang sama dengan baris lain.
  \item Baris ke-20: menghapus \texttt{pointer} ke berkas yang telah selesai digunakan.
  \item Baris ke-21: mencetak informasi
  \item Baris ke-22 \& 23: mencetak informasi statistik (jumlah data per kelas dan proporsinya dalam \%). Baris ke-22 digunakan untuk membuat perulangan berdasarkan jumlah kelas yang dalam contoh ini adalah jumlah elem dari variabel \texttt{jenis}. Sedangkan baris ke-23 digunakan untuk mencetak informasi berikut:
  \begin{itemize}
     \item nama kelas yang direpresentasikan melalui variabel \texttt{i}.
     \item jumlah data untuk kelas tertentu yang direpresentasikan melalui variabel \texttt{jenis[i]}. Karena jenisnya \texttt{int}, maka perlu diubah ke \texttt{str}.
     \item proporsi jumlah data, dilakukan dengan cara melakukan operasi pembagian jumlah data per kelas terhadap total data. Hasilnya ditampilkan dengan format dua angka desimal (\texttt{:4.2f}).
   \end{itemize} 
   Hasil dari menampilkan informasi statistik tersebut ditampilkan di \figurename~\ref{fig:iristat}.
\end{enumerate}

\begin{figure}
  \begin{center}
    \includegraphics[scale=2]{pics/iristat.png}
    \caption{Informasi statistik dataset iris}
    \label{fig:iristat}
  \end{center}
\end{figure}

\section{Berkas jamak}
Selanjutnya, sebagai latihan kedua, dataset yang akan digunakan adalah dataset tentang keamanan siber\footnote{\url{https://www.unsw.adfa.edu.au/unsw-canberra-cyber/cybersecurity/ADFA-NB15-Datasets/bot\_iot.php}}. Untuk seluruh dataset, terdapat 74 berkas yang berbeda. Tetapi, berkas pertamanya kosong sehingga akan diabakan dalam pembacaan. Perhatikan \lstlistingname~\ref{lst:catexplore}.

\lstinputlisting[language=python, numbers=left, numberstyle=\small, caption=Menghitung total data untuk setiap kelas, showstringspaces=false, label=lst:catexplore]{script/categoryExplore.py}

\section{Latihan}
\subsection{Berkas tunggal}
\lstlistingname~\ref{lst:bacairis} dibuat dengan asumsi bahwa dataset memiliki karakteristik sebagai berikut.
\begin{itemize}
  \item Tidak memiliki informasi kolom (\textit{column header}).
  \item Memiliki 5 kolom, di mana kolom ke-5 adalah kelas data.
\end{itemize}
Modifikasi \lstlistingname~\ref{lst:bacairis} agar kita tidak perlu membuat asumsi tersebut. Maksudnya, dengan atau tanpa \textit{column header}, tanpa pengetahuan jumlah kolom adalah 5, kita dapat membaca berkas dan menampilkan informasi seperti yang diminta.

\subsection{Berkas jamak}
Dataset\footnote{\url{https://www.unsw.adfa.edu.au/unsw-canberra-cyber/cybersecurity/ADFA-NB15-Datasets/bot\_iot.php}} menyediakan informasi sub kelas. Sebagai contoh, kelas \texttt{UDP} memiliki sub kelas \texttt{Dos} dan \texttt{DDoS}. Sementara \lstlistingname~\ref{lst:catexplore} hanya menampilkan informasi jumlah data per kelas. Karena itu, modifikasi \lstlistingname~\ref{lst:catexplore} agar dapat secara otomatis menghitung jumlah data per kelas sekaligus per sub kelas dan menampilkannya di layar.
