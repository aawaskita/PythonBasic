%-----------------------------------------------------------------------------%
\chapter*{Kata Pengantar}
%-----------------------------------------------------------------------------%
Dengan berkembang pesatnya keilmuan data (\textit{data science}), mahasiswa dan dosen perlu menguasai \textit{tools} yang dapat menunjang aktifitas mereka untuk mengeksporasi keilmuan tersebut. Salah satu \textit{tools} yang umum digunakan dalam keilmuan data berbasis pemrograman python. Karena alasan tersebut, buku elektronik ini disusun.

Secara umum, diktat ini dibagi ke dalam bagian pendahuluan yang membahas tentang sejarah singkat Python yang dilanjutkan ke bagian instalasi. Instalasi ini, meskipun sangat sederhana, terutama pada sistem operasi Linux, dapat menjadi sangat merepotkan bagi beberapa mahasiswa, terutama ketika mereka menggunakan sistem operasi Windows. Karena itu, instalasi akan dilakukan di sistem operasi Windows. Bagian selanjutnya adalah dasar-dasar pemrograman Python, terutama struktur data (\texttt{list}, \texttt{tuple} dan \texttt{dictionary}), interaksi dengan \textit{file} dan basis data, hingga membuat modul yang dapat digunakan kembali. Akhirnya, selamat mencoba pengalaman baru. 

\vspace*{2cm}
\begin{flushright}
\selectlanguage{bahasai}
Serpong, \today\\[0.1cm]
\vspace*{1cm}
Dr. Arya Adhyaksa Waskita

\end{flushright}
